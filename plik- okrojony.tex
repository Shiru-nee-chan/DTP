\documentclass{article}
\usepackage[utf8]{inputenc}

\usepackage{graphicx} 
\usepackage{fancyhdr}
\usepackage{polski}


\title{Moje ulubione hobby}  
\author{Sylwia Bomastyk}      
\date{26 Luty 2022r.}      
\newcommand{\ip}[2]{(#1, #2)}


\begin{document}             





\input{titlepage}
{\includegraphics {stos.png}}
\tableofcontents
\pagebreak

\section{Rozdział 1: Fantasy}

\subsection{Moimi ulubionymi książkami z gatunku fantasy są:}

1. Seria Dary Anioła - Cassandry Clare - Być świadkiem morderstwa – to przeżycie, które pozostawia w psychice ślad na całe życie. Clary Fray wiedze życie normalnej, nieco zbuntowanej nastolatki. Gdy jednak pewnego razu w klubie staje się świadkiem bezwzględnego morderstwa popełnionego przez trzech nastolatków na ich rówieśniku, jej dotychczasowa rzeczywistość kompletnie się rozpada.

{\includegraphics {dary.png}}

\begin{table}
    \begin{tabular}{|c|c|}
    \hline
		 Miasto niebiańskiego ognia & {\includegraphics {dary6.png}} \\
	\hline	 
    \end{tabular}
    \caption {6 części serii Dary Anioła}
\end{table}

2. Seria Baśniobór Brandona Mulla - Kendra i Seth mają spędzić wakacje u dziadka. Nie są tym zachwyceni. Dziadek wita ich całą serią zakazów i przestróg. Okazuje się jednak, że ten dziwny staruszek pełni bardzo ważną funkcję - jest strażnikiem niezwykłej krainy, Baśnioboru. Przez wieki Baśniobór był schronieniem czarodziejskich istot, żyjących według pradawnych reguł. Do dziś zamieszkują go w zgodzie zachłanne trolle, figlarne satyry, zgryźliwe czarownice, psotne chochliki i zazdrosne wróżki. Dzieci wcale nie zamierzają słuchać dziadka, który zabrania im zbliżać się do lasu. Nie wiedzą, że łamiąc zakaz, uwolnią siły zła, którym będzie trzeba stawić czoło. By uratować rodzinę, Baśniobór, a może nawet cały świat, Kendrze przyjdzie uczynić to, czego zawsze obawiała się najbardziej...

{\includegraphics {basniobor.png}}

\begin{table}
    \begin{tabular}{|c|c|}
    \hline
         Baśniobór & {\includegraphics {basniobor1.png}} \\
    \hline
    \end{tabular}
    \caption {5 części serii Baśniobór}
\end{table}

3. Seria Wodospady Cienia C.C. Hunter - Kylie Galen przechodzi trudny okres w życiu. Porzucona przez chłopaka dziewczyna przeżywa śmierć ukochanej babci i rozstanie rodziców, a na dodatek widuje dziwną postać, której nie dostrzega nikt inny… Pewien wieczór i nieodpowiednia impreza z nieodpowiednimi ludźmi zmienią życie Kylie na zawsze. Dziewczyna trafi na obóz dla trudnej młodzieży do Wodospadów Cienia. Szybko przekona się, że jej towarzysze to nie zwykła "trudna" młodzież, ale wilkołaki, wampiry, czarownice, elfy, zmiennokształtni i inne nastolatki o nadzwyczajnych zdolnościach. Sytuację dodatkowo skomplikuje pojawienie się Dereka i Lucasa, pólelfa i wilkołaka, wobec których Kylie nie pozostanie obojętna..

{\includegraphics {wodospady.png}}

\begin{table}
    \begin{tabular}{|c|c|}
    \hline
         Urodzona o północy & {\includegraphics {wodospady1.png}} \\
    \hline
    \end{tabular}
    \caption {5 części serii Wodospady cienia}
\end{table}

\section{Rozdział 2: Przygodowe}

\subsection{Moimi ulubionymi książkami z gatunku przygodowe są:}

1. Seria Zwiadowcy Johna Flannagana - Bohaterem książki jest piętnastoletni sierota, Will, wychowanek sierocińca. Jego ojciec poniósł bohaterską śmierć w trakcie ostatniego starcia ze złym baronem Morgarathem. Syn chciałby kontynuować tradycję. Może niekoniecznie dać się zabić, ale zostać rycerzem… Honor! Chwała! Odwaga!
Problem w tym, że do Szkoły Wojowników nie przyjmują mikrego wzrostu chucherek, które regularnie zbierają cięgi od rówieśników. A Will, cóż, nie dość, że niski, jest na dodatek chudy. Za to szybko ucieka.
Już wydaje się, że młodzieńcowi przypadnie mało atrakcyjna rola pańszczyźnianego chłopa, gdy nagle na scenie pojawia się tajemniczy Halt – zwiadowca. I on jeden spośród wszystkich mistrzów różnych szkół przyjmie Willa na ucznia. Zgryźliwy, humorzasty i wymagający, da chłopakowi popalić, ale… Wszystko jest lepsze od wiejskiego mozołu?

{\includegraphics {zwiadowcy.png}}

\begin{table}
    \begin{tabular}{|c|c|}
    \hline
         Ruiny Gorlanu & {\includegraphics {zwiadowcy1.png}} \\
	\hline	
    \end{tabular}
    \caption {16 części serii Zwiadowcy}
\end{table}

2. Trylogia Żelazny kruk Rafała Dębskiego - Pewnej nocy nad wioską, w której mieszka Evah rozlega się potężny głos, przemawiający w nieznanym języku. Słychać szum i hałas, zrywa się mocny wiatr. Mieszkańcy od razu odgadują, że to musi być Żelazny Kruk. Ojciec Evaha ukrywa chłopca w skrytce pod podłogą, w której zazwyczaj trzyma skóry upolowanych zwierząt, chowając je przed wzrokiem chciwych poborców. Po chwili umieszcza tam też młodszą córeczkę, jeszcze niemowlę. Evah słyszy hałasy, krzyki, tupot podkutych butów, dziwne warczenie. Modli się, żeby mała nie zaczęła płakać. Kiedy wszystko cichnie, wychodzi z kryjówki. Przed domem leży ciało jego ojca, który próbował bronić rodziny. Domy płoną, widać więcej trupów. Nigdzie jednak nie znajduje matki. Nie ma też większości mieszkańców wioski, w tym Verilli, sympatii chłopca.
Evah grzebie ojca i na jego grobie przysięga, że odszuka gniazdo Żelaznego Kruka, aby wyrwać bliskich z jego pazurów. Wyrusza w drogę z maleństwem na ręku. W najbliższej wsi zostawia siostrę pod opieką dalekiej rodziny. Krewni chcą, aby i on został, ale chłopiec zamierza zrobić wszystko, by ocalić matkę i kogo się tylko da.

{\includegraphics {kruk.png}}

\begin{table}
    \begin{tabular}{|c|c|}
    \hline
         Wyprawa & {\includegraphics {kruk1.png}} \\
    \hline
    \end{tabular}
    \caption {Trylogia Żelazny kruk}
\end{table}

3. Seria Percy Jackson i bogowie olimpijscy Ricka Riordana - Co by było, gdyby olimpijscy bogowie żyli w XXI wieku? Co by było, gdyby nadal zakochiwali się w śmiertelnikach i śmiertelniczkach i mieli z nimi dzieci, z których mogliby wyrosnąć wielcy herosi – jak Tezeusz, Jazon czy Herakles? Jak to jest – być takim dzieckiem? To właśnie przydarzyło się dwunastoletniemu Percy’emu Jacksonowi, który zaraz po tym, jak dowiedział się prawdy, wyruszył w niezwykle niebezpieczną misję.
Z pomocą satyra i córki Ateny Percy odbędzie podróż przez całe Stany Zjednoczone, żeby schwytać złodzieja, który ukradł przedwieczną „broń masowego rażenia” – należący do Zeusa piorun piorunów. Po drodze zmierzy się z zastępami mitologicznych potworów, których zadaniem jest go powstrzymać. A przede wszystkim będzie musiał stawić czoła ojcu, którego nigdy wcześniej nie spotkał, oraz przepowiedni, która ostrzegła go przed…

{\includegraphics {percy.png}}

\begin{table}
    \begin{tabular}{|c|c|}
    \hline
         Złodziej pioruna & {\includegraphics {percy1.png}} \\
    \hline
    \end{tabular}
    \caption {5 części serii Percy Jackson i bogowie olimpijscy}
\end{table}

\section{Rozdział 1: Fantasy}

\subsection{Moimi ulubionymi książkami z gatunku fantastyka naukowa są:}

1. Trylogia Kosogłos Suzanne Collins - Na ruinach dawnej Ameryki Północnej rozciąga się państwo Panem, z imponującym Kapitolem otoczonym przez dwanaście dystryktów. Okrutne władze stolicy zmuszają podległe sobie rejony do składania upiornej daniny. Raz w roku każdy dystrykt musi dostarczyć chłopca i dziewczynę między dwunastym a osiemnastym rokiem życia, by wzięli udział w Głodowych Igrzyskach, turnieju na śmierć i życie, transmitowanym na żywo przez telewizję. Bohaterką, a jednocześnie narratorką książki jest szesnastoletnia Katniss Everdeen, która mieszka z matką i młodszą siostrą w jednym z najbiedniejszych dystryktów nowego państwa. Katniss po śmierci ojca jest głową rodziny - musi troszczyć się o młodszą siostrę i chorą matkę, a jest to prawdziwa walka o przetrwanie...

{\includegraphics {igrzyska.png}}

\begin{table}
    \begin{tabular}{|c|c|}
    \hline
         Igrzyska śmierci & {\includegraphics {igrzyska1.png}} \\
	\hline	 
    \end{tabular}
    \caption {Trylogia Igrzyska śmierci}
\end{table}

2. Felix Net i Nika Rafała Kosika - Fantastyczna i przezabawna powieść o polskich trzynastolatkach z pewnego warszawskiego gimnazjum, nie tylko dla trzynastolatków!
Przeczytajcie o przyjaźni, zwariowanych wynalazkach, sztucznej inteligencji, skarbach, duchach, robotach, latającym talerzu i... nastoletnich uczuciach.
Relacje bohaterów z rówieśnikami, nauczycielami i rodzicami, polskie realia oraz żywy język sprawią, że książka ta stanie się ulubioną lekturą - nie będziecie się mogli oderwać!

{\includegraphics {fnin.png}}

\begin{table}
    \begin{tabular}{|c|c|}
    \hline
         Felix, Net i Nika oraz Gang Niewidzialnych Ludzi & {\includegraphics {fnin1.png}} \\
	\hline	
    \end{tabular}
    \caption {15 części serii Felix, Net i Nika}
\end{table}

3. Intruz Stephenie Meyer - Świat został opanowany przez niewidzialnego wroga. Najeźdźcy przejęli ludzkie ciała oraz umysły i wiodą w nich normalne życie. Jedną z ostatnich niezasiedlonych, wolnych istot ludzkich jest Melanie. Wpada jednak w ręce wroga, a w jej ciele zostaje umieszczona dusza o imieniu Wagabunda. Intruz bada myśli poprzedniej właścicielki ciała w poszukiwaniu śladów prowadzących do reszty rebeliantów...

{\includegraphics {intruz.png}}

\begin{table}
    \begin{tabular}{|c|c|}
    \hline
         Intruz & {\includegraphics {intruz1.png}} \\
    \hline
    \end{tabular}
    \caption {Intruz}
\end{table}

\section{Rozdział 4: Romans}

\subsection{Moimi ulubionymi książkami z gatunku romans są:}

1. 5 sekund do IO Małgorzaty Wardy - Akcja książki toczy się na dwóch planach – w realnym życiu i w grze, do której wchodzi szesnastoletnia bohaterka Mika. W szkole Miki dochodzi do tragicznej w skutkach strzelaniny. Tylko ona widziała sprawcę. Okazuje się, że dramat w liceum mógł mieć coś wspólnego z kontrowersyjną i tajemniczą grą, stworzoną w najnowszej technologii, gdzie gracz odczuwa nie tylko temperaturę otoczenia, zapachy i smaki, ale też ból. Może tam nawet przeżyć własną śmierć. Na prośbę policji Mika wkracza w wirtualny świat stworzony na wulkanicznym księżycu Io. Czy i ona wpadnie w pułapkę piekielnie niebezpiecznej gry?

{\includegraphics {io.png}}

\begin{table}
    \begin{tabular}{|c|c|}
    \hline
         5 sekund do IO & {\includegraphics {io1.png}} \\
	\hline	 
    \end{tabular}
    \caption {2 części serii 5 sekund do IO}
\end{table}

2. Akademia dobra i zła Soman Chainani - Śliczna, pamiętająca o dobrych uczynkach Sofia wydaje się urodzoną kandydatką na księżniczkę. Agata, dziwaczna nieładna, jest przez wszystkich uważana za wiedźmę.
Gdy więc obie dziewczynki zostają porwane do bliźniaczych szkół kształcących dobrych i złych bohaterów baśni, wydaje się jasne, która z nich gdzie trafi.
Dlaczego więc następuje pomyłka? I czy na pewno jest to pomyłka?

{\includegraphics {akademia.png}}

\begin{table}
    \begin{tabular}{|c|c|}
    \hline
         Akademia dobra i zła & {\includegraphics {akademia1.png}} \\
    \hline
    \end{tabular}
    \caption {6 części serii Akademia dobra i zła}
\end{table}

3. Trylogia czasu Kerstin Gier - Gwen to całkiem przeciętna nastolatka, choć pochodzi z nieprzeciętnej rodziny. Jej kuzynka Charlotta (przemądrzała zołza) jest nosicielką genu, dzięki któremu w wieku szesnastu lat zacznie podróżować w czasie. Ale czyżby Newton pomylił się w obliczeniach? Niespodziewanie bowiem to Gwen przenosi się w czasie, a nie Charlotta! Dziewczynę obejmuje opieką prastara tajna organizacja, która ukrywa wiele tajemnic. A to nie koniec niespodzianek. Jest jeszcze jeden podróżnik w czasie – chłopak. Bardzo przystojny chłopak. Ale jak tu się zakochać, gdy los rzuca się codziennie do innej epoki?

{\includegraphics {czas.png}}

\begin{table}
    \begin{tabular}{|c|c|}
    \hline
         Czerwień rubinu & {\includegraphics {czas1.png}} \\
    \hline
    \end{tabular}
    \caption {Trylogia czasu}
\end{table}

\section{Spis treści}
\begin{enumerate}
    \item Rozdział 1: Fantasy

    \item Rozdział 2: Przygodowe

    \item Rozdział 3: Fantastyka naukowa
	
	\item Rozdział 4: Romans

\end{enumerate}

\section{Spis tabel}
\begin{enumerate}
    \item Tab. 1 “6 części serii Dary Anioła”

    \item Tab. 2 “5 części serii Baśniobór”

    \item Tab. 3 “6 części serii Wodospady cienia”
	
	\item Tab. 4 “16 części serii Zwiadowcy”
	
	\item Tab. 5 “Trylogia Żelazny kruk”
	
    \item Tab. 6 “6 części serii Percy Jackson i bogowie olimpijscy”

    \item Tab. 7 “Trylogia Igrzyska śmierci”
	
	\item Tab. 8 “15 części serii Felix, Net i Nika”
	
	\item Tab. 9 “Intruz”

    \item Tab. 10 “2 części serii 5 sekund do IO”

    \item Tab. 11 “6 części serii Akademia dobra i zła”
	
	\item Tab. 12 “Trylogia czasu”

\end{enumerate}

\section{Spis obrazów}
\begin{enumerate}
    \item Rys. 1 “Części serii Dary Anioła”

    \item Rys. 2  “Części serii Baśniobór”

    \item Rys. 3 “Części serii Wodospady cienia”
	
	\item Rys. 4 “Części serii Zwiadowcy”
	
	\item Rys. 5 “Trylogia Żelazny kruk”

    \item Rys. 6 “Części serii Percy Jackson i bogowie olimpijscy”

    \item Rys. 7 “Trylogia Kosogłos”
	
	\item Rys. 8 “Części serii Felix Net i Nika”
	
	\item Rys. 9 “Intruz”

    \item Rys. 10 “Części 5 sekund do IO”

    \item Rys. 11 “Części Akademia dobra i zła”
	
	\item Rys. 12 “Trylogia czasu”

\end{enumerate}

\section{Spis literatury}
\begin{enumerate}
    \item Dary Anioła - Cassandra Clare

    \item Baśniobór - Brandon Mull

    \item Wodospady Cienia - C.C. Hunter
	
	\item Zwiadowcy - John Flannagan
	
	\item Żelazny kruk - Rafał Dębski

    \item Percy Jackson i bogowie olimpijscy - Rick Riordan

    \item Igrzyska śmierci - Suzanne Collins
	
	\item Felix, Net i Nika - Rafał Kosik
	
	\item Intruz - Stephenie Meyer

    \item 5 sekund do IO - Małgorzata Warda

    \item Akademia dobra i zła - Soman Chainani
	
	\item Trylogia czasu - Kerstin Gier

\end{enumerate}

\section{Źródła grafik i obrazów}
\begin{enumerate}
    \item empik.com 

    \item allegro.pl

\end{enumerate}

\end{document}
